\section{Linear Algebra} % (fold)

\begin{definition}
  Matrix is a collection of numbers
  \begin{displaymath}
    \begin{bmatrix}
      1 & 2 & 3 \\
      4 & 5 & 6 \\
      7 & 8 & 9
    \end{bmatrix}
  \end{displaymath}
\end{definition}

We can use matrices to index the rows as stocks and columns as dates. For
example:
\[
\begin{blockarray}{cccc}
    & \text{4/20} & \text{4/21} & \text{4/22} \\
    \begin{block}{c[ccc]}
        \text{AAPL} & \dots & \dots & \dots \\
        \text{NKE} & \dots & \dots & \dots \\
        \text{COIN} & \dots & \dots & \dots \\
    \end{block}
\end{blockarray}
\]

From a theoretical point of view, a matrix $A$ of size $m * n$ is an operator
that defines a linear transformation $$A : \bbR^m \rightarrow \bbR^n$$

\subsection{Eignvalue and Eigenvectors}

\begin{definition}
  Given an $\text{n} \times \text{n}$ matrix \textbf{$A$}, we define \textbf{$\lambda$} to be the
  \textbf{eigenvalue} and vector \textbf{$\vv$} to be the \textbf{eigenvector}
  such that:
  \begin{displaymath}
    A\vv = \lambda\vv
  \end{displaymath}
  We say $\vv$ is an \textbf{eigenvector} corresponding to $\lambda$.
\end{definition}

Do all matrices have eigenvalues?

\begin{align*}
  A\vv = \lambda\vv \Rightarrow (A-\lambda I)\vv &= 0 \\
  \text{\textit{iff} } det(A-\lambda I) &= 0
\end{align*}

Where $det(A-\lambda I)$ is a polynomial of degreen $n$ in terms of $\lambda$
which has a solution.

\subsection{Geometrical meaning}

\begin{center}
\usetikzlibrary {3d}
\begin{tikzpicture}[->]
  \draw (0,0,0) -- (xyz cs:x=2);
  \draw (0,0,0) -- (xyz cs:y=2);
  \draw (0,0,0) -- (xyz cs:z=2);
  \draw[red] (0,0,0) -- (1.25, 1.0, 1.0) node[above] {$\vv_1$};
  \node at (2.75,3,3) {$\mathbb{R}^3$};

  \draw[->,bend left=45] (3,2,2) to node[midway,above] {$A$} (5,2,2);

  \draw (5,0,0) -- (xyz cs:x=7);
  \draw (5,0,0) -- (xyz cs:y=2,x=5);
  \draw (5,0,0) -- (xyz cs:z=2,x=5);
  \draw[red] (5,0,0) -- (6.875, 1.5, 1.5) node[above] {$\lambda\vv_1$};
  \node at (8, 3, 3) {$\mathbb{R}^3$};
\end{tikzpicture}
\end{center}

Suppose $A$ is a $3 \times 3$ matrix. There exists some vector $\vv$ which when
$A\vv$, the result is just $\vv$ scaled by a constant factor $\lambda$.

\begin{definition}
  An $n \times n$ matrix $A$ is \textbf{diagonalizable} if there exists an
  orthonormal\footnote{An orthonormal matrix is defined as $U U^{-1} = I$}
  matrix $U$ s.t. 
  \begin{displaymath}
    A = U D U^{-1}
  \end{displaymath}
  for a diagnoal matrix D.
\end{definition}

A \textbf{diagnoal matrix} has all it's non zero values on the diagnoal as
shown below:

\[
  \begin{bmatrix}
    a_{11} & 0 & 0 & \dots & 0 \\
    0 & a_{22} & 0 & \dots & 0 \\
    0 & 0 & \ddots & \ddots & \vdots \\
    \vdots & \vdots & \ddots & a_{n-1,n-1} & 0 \\
    0 & 0 & \dots & 0 & a_{nn} \\
  \end{bmatrix}
\]


% section Linear Algebra (end)
