\section{Introduction, Financial Terms and Concepts} % (fold)

Introductory lecture giving an overview of topics covered in the course. Opens
with an anecdote of Dr. Choongbum Lee speaking about his experience starting
at the Morgan Stanley options desk coming from Solomon. Morgan Stanley calls
Vega\footnote{Vega or kappa is a measure of volitility of an underlying asset}
Kappa. There was a joke that the less educated people at solomon brothers call
kappa Vega.

Modern investment banks are structured into the following groups:

\begin{itemize}
  \item \textbf{Fixed income}: debt and debt derivative products
  \item \textbf{Equities}: stocks and stock derivatives (ie. ETFs, indicies, etc)
  \item \textbf{Investment banking}: corporate finance, M\&A etc.
\end{itemize}

Why do we have financial markets? Connecting investors with borrowers. A dealer
won't actually be a counter-party in a trade, where a broker will execute the
trade as the counter-party. Hedge funds take advantage of market inneffeciencies
in order to find. The main active participants in the market are
\textbf{traders}. Traders involve three main types:

\begin{enumerate}
  \item Hedger
  \item Market Maker, bid and offer
  \item Proprietary trader, fund portfolio manager, beta and alpha
\end{enumerate}

Proprietary are trying to use active management strategies to beat the returns
in the market. 

\begin{equation} 
  R(\alpha) = \alpha + \beta *R(\beta)
\end{equation}

\subsection{Hedging}

Let's assume the interest rate in Japan is lower and the rate in Australia is
higher. In this case, it would be logical to borrow JPY and invest the
moeny in Australia. However this doesn't always work due to slippage between the
two currencies. If you wanted to hedge this transaction, then you would need to
pay AUD to recieve JPY using foreign exchange (\textbf{FX}) swaps and
cross-currency basis.

For example, consider Intel who may have a large portion of revenue in foreign
currency. They need to find ways to hedge the currency exposure for the assets
sitting in non USD accounts.

\subsection{Market Making}

Simple product bid/offer, price transparency, liquidity. Some of the "Greeks"
they track are:

\begin{itemize}
  \item \textbf{delta}:  Difference between position and underlying value of the
    asset.
  \item \textbf{gamma}: Derivative of Change of the portfolio 
  \item \textbf{theta}
\end{itemize}

\subsection{Proprietary Trading}

Directional trading, long/short. Also known as gut trading. Also utilize
arbitrage, which takes advantage of the difference between pricing in different
markets. They also can take advantage of the spot vs future value. Another
strategy is "value investing"\footnote{\url{https://en.wikipedia.org/wiki/
Value_investing}} which uses underlying principles to value a stock in a long
term time horizon.

\subsection{Financial Mathematics}

Math can assist with Pricing Models, Risk Management, and Trading Strateiges.

%; section Introduction, Financial Terms and Concepts (end)
